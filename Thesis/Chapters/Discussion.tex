\chapter*{Discussion}
\addcontentsline{toc}{chapter}{5. Discussion}
1500 words maximum, including citations

primary Auditory pathway (VERY simplified): cochlea $\rightarrow$ brainstem (ventral and dorsal cochlear nuclei) $\rightarrow$ midbrain (inferior colliculus) $\rightarrow$ thalamus (medial geniculate nucleus) $\rightarrow$ primary auditory cortex (A1 or Brodmann's areas 41)

Electrodes that look like they sit over A1 - \hl{TP8}, T8, FT8, CP6, C6, FC6 (right). \hl{TP7, T7}, FT7, CP5, \hl{C5}, FC5 (left)

Electrodes in dark side regions in figure - P7, \hl{TP7, T7, C5}, P8, \hl{TP8}

laterality: darker on the left side - possibly to do with lyrics and speech processing. 2/3 of songs either have lyrics or induce lyrics in the mind (participants said it was hard to not think of the lyrics). 

3rd bar: first bar sets up the template, second bar is a confirmation of the song's pattern, 3rd bar marker is confirmation of what they expect to hear? 

Why does our topography look so different from the one seen in Schaefer 2011? Her component describes similarities, ours describes differences. 

Imagination: we have determined what brain signals are important for classifying perception. We can now use this and apply it to data collected during music imagination.

Bring in applications of this BCI (patients) at the end of the discussion.