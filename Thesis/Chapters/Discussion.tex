\chapter*{Discussion}
\addcontentsline{toc}{chapter}{5. Discussion}
The goal of these experiments was to investigate whether music perception and imagination can be detected using EEG and therefore used to control a \ac{BCI}. 
\cite{schaefer_name_2011} were able to classify perceived music stimuli based on the unique shapes of principle component time courses, but we were unable to replicate this result. 
The most likely reason is the number of stimuli presented to participants. 
In the Schaefer study participants heard each stimulus more than a hundred times. 
In our experiment participants heard each stimulus five times. 
The small number of trials is likely also responsible for our inability to classify imagination using either the PCA technique or machine learning. 

Using machine learning techniques we were able to classify perception of music from the EEG signal. 
However, our technique failed when we applied it to the imagination of music. 
There are multiple reasons that could explain why we were unable to classify music imagination.
In music perception the timing of the music is consistent across trials (e.g. the second beat of the song always occurs at a consistent time point) because the timing is driven by the stimulus. 
In imagination this timing may fluctuate across trial and across participants, because after the end of the tempo cue there is no external stimulus.
A single participant may imagine music at a different rate on different trials and some participants may have a tendency to speed up throughout their imagination while others may have a tendency to slow down. 
This fluctuation in timing is likely the cause of our low imagination classification rates. 

The secondary goal of these experiments was to determine what neural processes drive our ability to classify music perception and imagination.
It is tempting to interpret the results of a neural network, but it is generally not possible to determine why a trained neural network makes a particular decision \cite{towell_1992_interpretation}.
The results of a neural network present prediction results and are not representative of the underlying process that relate input to output \cite{intrator_2001_interpreting}.
However, to investigate whether we could glean any brain-related information from our neural network (\autoref{fig:model_W}) we focused on whether the spatial or temporal filters can be related to any biological or musical characteristics.

The spatial filter in layer 1 indicates which electrodes carry the \ac{EEG} data important for classification. 
However, due to the nature of \ac{EEG} we are unable to comment on where the data from these electrodes is produced.
The auditory research literature is in general consensus on what principal components of auditory processing look like, but when the biologically produced fronto-central component maps are compared to the filter in layer 1 there are no similarities. 
To force the net to use biologically produced information we exchanged the neural net's first layer with the principal components calculated in \autoref{fig:components} (C). 
This resulted in a decrease in classification accuracy. The results from the neural net using biologically produced spatial maps can be seen in Appendix \ref{appendix:PCAInvestigation}.

The second layer of the neural net produced temporal filters that indicate time periods in the stimuli important for classification. 
Upon closer investigation there were no auditory characteristics that stood out as being unique to each of the highlighted time periods.
To determine whether the patterns in the filters are driven by a cognitive process we conducted a behavioural experiment.
The results showed that the highlighted time periods do not occur at the moment when participants recognize the piece of music. 
Based on these results we know what is \emph{not} responsible for highlighting these moments: the importance of these moments is not due to auditory characteristics of the stimuli or a moment of recognition.
At this time we are unable to say what is causing these time periods to be flagged as important for stimuli classification. 

Although we were able to classify music perception, we were not able to classify music imagination.
Future experiments should aim to disentangle what during perception is driving the classifier and to enhance this during imagination.
To do this we may need to take a step back and use simpler stimuli.
Rhythm is an inherent part of music, but on its own is a series of events at time intervals without information about tone, lyrics etc.
It is possible to classify the perception of rhythms \cite{stober2014audiomostly}, so capitalizing on rhythm's auditory simplicity may be an effective way to learn  what characteristics are necessary to drive a music-based \ac{BCI}.
It will also be important during future experiments to continue to cue participants to the tempo during imagination using a metronome.
This will ensure that all participants imagine at the same rate and are consistent across multiple trials.