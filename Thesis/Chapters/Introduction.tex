\chapter{Introduction}
%\addcontentsline{toc}{chapter}{1. Introduction}

Everybody imagines music. Imagining music can be defined as a deliberate internal recreation of the perceptual experience of listening to music \cite{schaefer_name_2011}.
Individuals can imagine themselves producing music, imagine listening to others produce music, or simply "hear" the music in their heads. 
Music imagination is used by musicians to memorize music and anyone who has ever had an "ear-worm" -- a tune stuck in their head -- has experienced imagining music. Because of its simplicity no training is required to imagine a song, and researchers have been investing the utility of music imagery for \acp{BCI}.
A \ac{BCI} is a system that allows an external device to be controlled or modified using brain activity. 
Music imagery appears to be a very promising means for driving \acp{BCI} that use \ac{EEG} -- a popular non-invasive neuroimaging technique that relies on electrodes placed on the scalp to measure the electrical activity of the brain.
For instance, Schaefer \etal\cite{schaefer_measuring_2011} argue that
\emph{``music is especially suitable to use here as (externally or internally generated) stimulus material, since it unfolds over time, and \ac{EEG} is especially precise in measuring the timing of a response.''}
For patients that have difficulties communicating behaviourally (e.g. patients with locked-in syndrome) \ac{BCI}s are a promising communication tool. 
{BCI}s that currently exist are generally binary systems that allow the user to choose between two options to answer yes/no questions. %limiting communication abilities. 
A system with a larger number of answer options would offer a more complete communication experience. 
Using music as the basis for a \ac{BCI} is a promising way to build such a system due to the large number of musical pieces that exist. 
Ideally, a music-based \ac{BCI} would allow the user to imagine a piece of music in order to convey a particular thought. 
However, the translation from music imagination will require careful processing of the EEG data. 

EEG data contain a variety of signals, elicited by sounds, that can be exploited by a \ac{BCI}. 
%However, the EEG data is full of unwanted signals (noise) and extracting the relevant information can be a challenge.
%Decoding brain wave recordings to classify different states is a relatively new field of research.
%A recent review of neuroimaging methods for \ac{MIR} that also covers techniques different from EEG is given in \cite{ismir2015kaneshiro}.
\ac{EEG} signals have been used to measure emotions induced by music perception \cite{lin_eeg_2009,cabredo_emotion_2012} and to distinguish perceived rhythmic stimuli \cite{stober2014nips}.
Further in the rhythmic domain it has been shown that oscillatory neural activity in the gamma frequency band (20-60 Hz) is sensitive to accented tones in a rhythmic sequence \cite{snyder_gamma-band_2005}.
Oscillations in the beta band (20-30 Hz) entrain to rhythmic sequences \cite{cirelli_beta_2014, merchant_beta_2015} and increase in anticipation of strong tones in a non-isochronous, rhythmic sequence \cite{iversen_top-down_2009,fujioka_beta_2009,fujioka_internalized_2012}.
The magnitude of \acp{SSEP}, which reflect neural oscillations entrained to the stimulus, changes when subjects hear rhythmic sequences for frequencies related to the metrical structure of the rhythm.
This is a sign of entrainment to beat and meter \cite{nozaradan_tagging_2011,nozaradan_selective_2012}. 
\ac{EEG} studies have further shown that perturbations of the rhythmic pattern lead to distinguishable \acp{ERP} \cite{geiser_early_2009}.
Furthermore, Vlek \etal \cite{vlek_shared_2011} showed that imagined auditory accents imposed on top of a steady metronome click can be recognized from EEG.

Recently studies have identified a close relationship between the brain areas that are active during the imagination and the perception of music \cite{halpern_fmri_2004,Kraemer2005,Herholz2008,herholz_2012}. 
Although not directly relevant to a \ac{BCI}, we can learn a lot from \ac{EEG} data collected while participants listen to melodies.
Exploring EEG data during music perception can inform how we approach music imagination data and the brain signals recorded while listening to music could serve as reference data to determine which salient elements are to be expected during imagination. 
By using perception data as a way to train our \ac{BCI} we can cut down on the amount of imagination training needed which will reduce potential user fatigue.

EEG has already successfully been used to classify perceived melodies. 
In a study by Schaefer \etal \cite{schaefer_name_2011}, 10 participants \textit{listened} to 7 short melody clips with a length between 3.26s and 4.36s.
For single-trial classification, each stimulus was presented 140 times in randomized back-to-back sequences of all stimuli.
Using a quadratically regularized linear logistic-regression classifier with 10-fold cross-validation, they were able to successfully classify the \acp{ERP} of single trials.
Within subjects, the accuracy varied between 25\% and 70\%.
Applying the same classification scheme across participants, they obtained between 35\% and 53\% accuracy.
%In a further analysis, they combined all trials from all subjects and stimuli into a grand average \ac{ERP}.
%Using singular-value decomposition, they obtained a fronto-central component that explained 23\% of the total signal variance.
%The time courses corresponding to this component showed significant differences between stimuli that were strong enough to allow cross-participant classification.
%As Hubbard concludes in his recent review of the literature on auditory imagery, \emph{``auditory imagery preserves many structural and temporal properties of auditory stimuli''} and \emph{``involves many of the same brain areas as auditory perception''} \cite{hubbard_auditory_2010}. 
%This is also underlined by Schaefer \cite[p. 142]{schaefer_measuring_2011} whose \emph{``most important conclusion is that there is a substantial amount of overlap between the two tasks} [music perception and imagination]\emph{, and that `internally' creating a perceptual experience uses functionalities of `normal' perception.''}

Brain activity during music imagination has also been detected by \ac{EEG} \cite{schaefer_shared_2013}, and encouraging preliminary results for recognizing imagined music fragments from \ac{EEG} recordings were reported in \cite{schaefer_single_2009} in which 4 out of 8 participants produced imagery that was classifiable (in a binary comparison) with an accuracy between 70\% and 90\% after 11 trials.

Although \ac{EEG} has been used to decode music imagination the accuracy levels are not robust enough for these decoding techniques to be used in a \ac{BCI}. 
This could be because the EEG processing methods may not be sensitive enough to the subtle changes that occur during music imagination. 
The sophisticated signal processing techniques used in machine learning can lend expertise to this challenge. 

Machine learning is a technique that uses algorithms that can learn from and make predictions about data.
For example, the programs used by postal services to recognize handwriting on envelopes or the speech recognition software in your cell phone are based on machine learning techniques.
One such technique is based on convolutional neural networks (\acp{CNN}).
\acp{CNN} were inspired by the powerfully complex visual system found in humans and other animals.
In the retina we have cells that are responsive to small regions of the visual field \cite{hubel_receptive_1963}. 
As information moves along the visual processing stream and into the brain single cells in higher layers receive input from multiple cells in lower layers.
At each level of this process more information is combined.
This gives cells higher up in the processing stream an increasingly global view of what information was collected by the retinal cells.
Complex visual information is processed farther along the processing stream than simple information as cells in these far layers are sent information from a larger number of retinal cells.
\acp{CNN} work in a similar way to process complex data. 
The processing units in a \ac{CNN} act like cells in the visual system.
The ``receptive field'' of each one of these units is determined by a filter.
Each filter is created based on a variety of parameters set by the researcher or determined by the network during the training process.
%In this experiment the filters used are \acp{CAE} (as described in \cite{masci_stacked_2011}).
%A \ac{CAE} is an artificial neural network that is able to extract features from data without supervision.
% feature extractor that scales well to high dimensional input.
%CAEs are a special variant of a \ac{CNN} that encode their input using convolution into a compressed internal representation. 
%This representation is then decoded using de-convolution into the original space while trying to minimize the reconstruction error. 
%This type of neural network can learn a meaningful but compressed representation of the data.
%Importantly, \acp{CAE} preserve spatial information allowing them to learn biologically plausible features.%
%Once the \ac{CAE} has extracted the important features from the training data and constructed the filters these filters can be used in a \ac{CNN}.
%These filters are very similar to the components produced in \ac{PCA} and \ac{ICA} with the difference being the parameters used to compute the components. 
%The \ac{CAE} filters aim to minimize the \ac{MSRE} rather than the orthogonality or independence criteria used in \ac{PCA} and \ac{ICA} respectively. 
Before a \ac{CNN} can be used to classify data it must learn the characteristics of the data. 
The \ac{CNN} is trained using a subset of the data and the filters are optimized to produce the best classification results. 
The optimized filters are applied to new data and the accuracy of the classification is determined. 

%SUMMARY OF WHAT WE TRIED AND THE TECHNIQUES WE USED
%Using this analysis technique we aim to be able to classify what a participant was listening to or imagining based on the signal in their EEG data alone. 
To approach this problem we first tried an ERP analysis similar to that of Schaefer \etal\cite{schaefer_name_2011} in order to determine which piece of music a participant was listening to or imagining.
When this proved unsuccessful we used a machine learning technique called deep learning to detect more subtle differences in the EEG that would be more useful in classifying stimuli. 
Using this technique we were able to classify perception of 12 music pieces with a 28.7\% accuracy (chance=8.3\%).
We were unable to accurately classify imagination of music \hl{specific values go here}.